\documentclass[pdftex,11pt,a4paper]{article}

% The following makes latex use nicer postscript fonts.

\usepackage[english]{babel}
\usepackage[colorlinks,urlcolor=blue,linkcolor=blue]{hyperref}
\usepackage{fancyvrb}
\usepackage{relsize}
\usepackage{caption} 
\usepackage{pdfpages}
\usepackage{subfig}
\usepackage{enumitem}

\usepackage{vubtitlepage}
\author{Oliver Verhaegen - 87116\\
overhaeg@vub.ac.be}
\title{Capita Selecta of Software Engineering: Project}

%\promotortitle{Promotor/Promotors}
\promotor{Maja D'Hondt}
%\advisors{Steven Adriaensen}
\faculty{Faculty of Science}
\promotortitle{Teacher}
%\advisortitle{Teaching Assistant}
\department{Department of Computer Science}
\reason{Second deliverable}

\date{\today}

\begin{document}
% Then english TitlePage
\maketitlepage

\tableofcontents
\clearpage

\section{Source Code Repository}

The source code for this project is avalaible at \url{https://github.com/overhaeg/CSPL-Project}.

\section{Project Description}

The goal of this project is the implementation of an interpreter of System F$\omega$. System F$\omega$ is an extension of System F, which is itself an extension of 
the Simply Typed Lambda Calculus (STLC). STLC introduces typing to lambda calculus, with a single function operator. 
System F extends STLC by adding support for parametric polymorphism, and System F$\omega$ extends this further with higher-order polymorphism. 
The interpreter will require a Parser, a Type checker and an Evaluator.\\
\noindent
System F$\omega$ will be implemented using the Haskell language, and should therefore be runnable on every system for which an implementation of the Glasgow Haskell Compiler (GHC) exists.
To help with the implementation of the parser we will use the Happy Parser Generator for Haskell.\\
This interpreter was implemented as a three-part project for the course Capita Selecta of Programming Languages. 



\end{document}
